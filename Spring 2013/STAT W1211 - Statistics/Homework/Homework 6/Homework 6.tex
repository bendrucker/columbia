\documentclass{article}

%%%%%% Include Packages %%%%%%
\usepackage{sectsty}
\usepackage{amsmath,amsfonts,amsthm,amssymb}
\usepackage{fancyhdr}
\usepackage{lastpage}
\usepackage{setspace}
\usepackage{graphicx}

%%%%%% Formatting Modifications %%%%%%

\usepackage[margin=2.5cm]{geometry} %% Set margins
\sectionfont{\sectionrule{0pt}{0pt}{-8pt}{0.8pt}} %% Underscore section headers
\setstretch{1.2} %% Set 1.2 spacing

%%%%%% Set Homework Variables %%%%%%

\newcommand{\hwkNum}{6}
\newcommand{\hwkAuthors}{Ben Drucker}

%%%%%% Set Header/Footer %%%%%%

\pagestyle{fancy} 
\lhead{\hwkAuthors} 
\rhead{Homework \#\hwkNum}
\rfoot{\textit{\footnotesize{\thepage /\pageref{LastPage}}}}
\cfoot{}
\renewcommand\headrulewidth{0.4pt}
\renewcommand\footrulewidth{0.4pt}

%%%%%% Document %%%%%%

\begin{document}

\title{Homework \#\hwkNum}
\author{\hwkAuthors}
\date{}

\maketitle

%%%%%% Begin Content %%%%%%

\section*{5.1}
	\subsection*{12}
		\subsubsection*{a)}
			\begin{align*}
				P(X>3) &= \int_3^\infty \int_0^\infty f(x,y)dy dx \\
				&= \int_3^\infty \int_0^\infty xe^{-x(1+y)}dydx \\
				&= \int_3^\infty xe^{-x}\int_0^\infty e^{-xy}dydx \\
				&= \int_3^\infty xe^{-x} \left [\frac{e^{-xy}}{-x} \right ]^\infty_0 dx \\
				&= \int_3^\infty e^{-x} dx \\
				&= \left[ -e^{-x} \right ]_3^\infty \\
				&= .05
			\end{align*}
		\subsubsection*{b)}
			$X:$
				\begin{align*}
					f(x) &= \int_y f(x,y) dy \\
					&= \int_0^\infty xe^{-x(1+y)} dy \\
					&= xe^{-x} \int_0^\infty e^{-xy} dy \\
					&= xe^{-x} \left [ \frac{e^{-xy}}{-x} \right ]^\infty_0 \\
					&= e^{-x} \Leftrightarrow x\geq 0
				\end{align*}
			$Y:$
				\begin{align*}
					f(y) &= \int_x f(x,y) dx \\
					&= \int_0^\infty xe^{-x(1+y)} dx \\
					&= \left [ x \left ( \frac{e^{-x(1+y)}}{{-1-y}} \right ) - \frac{e^{-x(1+y)}}{(1+y)^2} \right ]_0^\infty \\
					&=\frac{1}{(1+y)^2} \Leftrightarrow y\geq 0
				\end{align*}
			No. The joint PDF is not a factor of the product of the marginal PDFs. 		
		\subsubsection{c)}
			$P(\max(X,Y) > 3) =1 - P(\max(X,Y) \leq 3) =1 - \int_0^3 xe^{-x} \left ( \int_0^3e^{-xy} dy \right ) dx = 1- \int_0^3 e^{-x} (1-e^{-3x} )dx = e^{-3} + 1 - \frac{1}{4} e^{-12} \approxeq .3$
	\subsection{16}
		\subsubsection{a)}
			\begin{align*}
				f(x_1,x_3) &= \int_{-\infty}^\infty f(x_1, x_2, x_3)dx_2 \\
				&= \int_0^{1-x_1-x_3} kx_1x_2(1-x_3)dx_2 \\
				&= 72x_1(1-x_3)(1-x_1-x_3)^2 \Leftrightarrow x_1,x_3 \geq 0; x_1+x_3 \leq 1. 
			\end{align*}
		\subsubsection*{b)}
			$P(X_1+X_3 \leq .5) = \int_0^5 \int_0^{5-x_1} 72x_1(1-x_3)(1-x_1-x_3)^2 dx_1 dx_2 = .5312.$ Calculated using Wolfram Alpha. 
		\subsubsection*{c)}
			$f_{x_1}(x_1) = \int_{- \infty}^\infty f(x_1, x_3)dx_3 = \int	72x_1(1-x_3)(1-x_1-x_3)^2 dx_3 = 18x_1 -48x_1^2+36x_1^3-6x^5_1 \Leftrightarrow 0 \leq x_1 \leq 1.$
\section*{5.2}
	\subsection*{22}
		\subsubsection*{a)}
			$$ E(X+Y) = \sum_x \sum_y (x+y)p(x,y,) = (0+0)(.02) + (0+5)(.06)+(5+0)(.04) + ... + (10+15)(.01) = 14.1$$
		\subsubsection*{b)}
			$$E(\max(X,Y) = \sum_x \sum_y \max(x+y)p(x,y) = 0*.02 + 5*.06 + 5*.04 + ... + 15*.01 =9.6$$
	\subsection*{32}
		??
	
	\subsection*{36)}
		??			 				 				
		
			
			
%%%%%% End Content %%%%%%      
\end{document}
