\documentclass{article}

%%%%%% Include Packages %%%%%%
\usepackage{sectsty}
\usepackage{amsmath,amsfonts,amsthm,amssymb}
\usepackage{fancyhdr}
\usepackage{lastpage}
\usepackage{setspace}
\usepackage{graphicx}
\usepackage{array}

%%%%%% Formatting Modifications %%%%%%

\usepackage[margin=2.5cm]{geometry} %% Set margins
\sectionfont{\sectionrule{0pt}{0pt}{-8pt}{0.8pt}} %% Underscore section headers
\setstretch{1.2} %% Set 1.2 spacing

%%%%%% Set Homework Variables %%%%%%

\newcommand{\hwkNum}{11}
\newcommand{\hwkAuthors}{Ben Drucker}

%%%%%% Set Header/Footer %%%%%%

\pagestyle{fancy} 
\lhead{\hwkAuthors} 
\rhead{Homework \#\hwkNum}
\rfoot{\textit{\footnotesize{\thepage /\pageref{LastPage}}}}
\cfoot{}
\renewcommand\headrulewidth{0.4pt}
\renewcommand\footrulewidth{0.4pt}

%%%%%% Document %%%%%%

\begin{document}

\title{Homework \#\hwkNum}
\author{\hwkAuthors}
\date{}

\maketitle

%%%%%% Begin Content %%%%%%

\section*{7.3}
	\subsection*{32}
		CI: $\overline{X} \pm t \frac{S}{\sqrt{n}} = 1584 \pm t_{19, .005} \frac{607}{\sqrt{20}} = 1584 \pm 2.861 \frac{607}{\sqrt{20}} = (1195.68, 1972.32)$
	\subsection*{34}
		\subsubsection*{a)}				
			$8.48-1.771 \frac{.79}{\sqrt{14}} = 8.11.$ With 95\% confidence, the true mean of all joints is in the interval (8.11, $\infty$). For an infinite number of sample confidence intervals, 95\% will include the true mean.  A normal distribution is assumed.
		\subsubsection*{b)}
			$8.48 - 1.771 * .79 \sqrt{1+\frac{1}{14}} = 7.03$. If we calculate this bound for an infinite number of samples, 95\% will give the lower bound for future values of a joint.
\section*{7 Supplement}
	\subsection*{50}
		$\overline{x} = \frac{229.764+233.502}{2} 231.63; t_{.025, 5-1} = 2.78\\
		233.502 - 229.764 = 3.74; 2*2.78 \frac{s}{\sqrt{5}} =s \rightarrow s = \frac{\sqrt{5}*3.74}{2*2.78} = 1.51\\
		t_{.005,4} = 4.604 \\
		231.63 \pm 4.604 \frac{1.51}{\sqrt{5}} = 213.63 \pm 3.1$
		
	\subsection*{60}			 
		$(z_Y+z_{\alpha - Y} \frac{s}{\sqrt{n}}) \rightarrow \min(z_Y+z_{\alpha - Y} \frac{s}{\sqrt{n}}) \rightarrow \min[\Phi^{-1}(1-Y) + \Phi^{-1} (1-\alpha + Y).$
		\\ Setting the derivative equal to 0:
		\\ $ \frac{1}{\Phi(1-Y)} = \frac{1}{\Phi(1-\alpha+Y)} \Rightarrow Y = \frac{\alpha}{2}$
\section*{8.1}
	\subsection*{2}
		\subsubsection*{a)}
			Yes.
		\subsubsection*{b)}
			No, because $H_0$ is not an equality claim.
		\subsubsection*{c)}
			No, because $H_a$ is the equality claim instead of $H_0$.
		\subsubsection*{d)}
			No. $\mu_1 -\mu_2$ should appear in $H_a$.
		\subsubsection*{e)}
			No because $S^2$ is a statistic and shouldn't be into a hypothesis.
		\subsubsection*{f)}
			No, both $H_0$ and $H_a$ can't be equality claims.
		\subsubsection*{g)}
			Yes
		\subsubsection*{h)}
			Yes
	\subsection*{4}	
		Versus $H_a: \mu < 5$. In this case, the type I error should be avoided at all costs whereas the type II error is not as serious.
	\subsection*{6}
		$H_0: \mu  = 40; H_a: \mu \not = 40. \mu \not = 40$ is interesting for the manufacturer in either direction. A type I error would be rejected a fuse where $\mu$ is actually 40. A type 2 error would be letting a fuse through where $\mu$ is 40.
				
				 
	 		
%%%%%% End Content %%%%%%      
\end{document}